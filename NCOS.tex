%\documentclass[12pt]{article}


\documentclass[12pt]{article}
\usepackage[backend=bibtex,natbib,style=numeric-comp,sorting=none, doi=false, isbn=false,url=false]{biblatex}
\usepackage{slashed}
\addbibresource{refs}
\input epsf.sty

\pdfoutput=1

% Use Chancery Font
\DeclareFontFamily{OT1}{pzc}{}
\DeclareFontShape{OT1}{pzc}{m}{it}{<-> s * [1.10] pzcmi7t}{}
\DeclareMathAlphabet{\mathpzc}{OT1}{pzc}{m}{it}
\newcommand{\josh}[1]{\textcolor{blue}{[Joshua: #1]}}
\newcommand{\cw}[1]{\textcolor{gray}{[Charles: #1]}}
%\usepackage{chngcntr}
%\counterwithout{equation}{section}



\newcommand{\bb}[1]{\mathbb{#1}}
\newcommand{\wt}[1]{\widetilde{#1}}
\newcommand{\ol}[1]{\overline{#1}}
\newcommand{\mc}[1]{\mathcal{#1}}
\newcommand{\hmch}{\hat{\mc{H}}}
\newcommand{\citeme}[1][]{{\color{red}[*#1]}}

\usepackage{stmaryrd}


\usepackage{draft}
\usepackage[weather]{ifsym}

\usepackage{hyperref}
\usepackage{graphicx,color,subfig}
\usepackage{skak}
\usepackage{empheq}
\usepackage{tikz}
\usepackage{bbm}

% Use Chancery Font
\DeclareFontFamily{OT1}{pzc}{}
\DeclareFontShape{OT1}{pzc}{m}{it}{<-> s * [1.10] pzcmi7t}{}
\DeclareMathAlphabet{\mathpzc}{OT1}{pzc}{m}{it}

\usetikzlibrary{calc}
\usetikzlibrary{snakes}
\usetikzlibrary{arrows.meta}
\usetikzlibrary{decorations.pathmorphing}
\usetikzlibrary{decorations.markings}
\usetikzlibrary{bending}
\tikzset{snake it/.style={decorate, decoration=snake}}
\usetikzlibrary{shapes.misc}
\tikzset{cross/.style={cross out, draw=black, minimum size=2*(#1-\pgflinewidth), inner sep=0pt, outer sep=0pt},
%default radius will be 1pt.
cross/.default={1pt}}

\usepackage[T1]{fontenc}
\usepackage{esint}
\usepackage{lmodern}

\newcommand{\fixme}[1]{{\bf {\color{red}[#1]}}}
\newcommand{\vv}[1]{\left\langle #1 \right\rangle}
\newcommand{\un}[1]{\underline{#1}}
\newcommand{\BL}[1]{{ {\color{blue}[#1]}}}

\newcommand{\bgcom}[1]{\fixme{BG: #1}}

\def\be#1\ee{\begin{align}#1\end{align}}
\newcommand\nn{\nonumber}

\newcommand{\calA}{\mathcal{A}}
\newcommand{\bz}{\bar{z}}


\definecolor{dark green}{rgb}{0.7,1,0.64}

\usepackage{listings}
\usepackage{xcolor}

\definecolor{codegreen}{rgb}{0,0.6,0}
\definecolor{codegray}{rgb}{0.5,0.5,0.5}
\definecolor{codepurple}{rgb}{0.58,0,0.82}
\definecolor{backcolour}{rgb}{0.95,0.95,0.92}

\lstdefinestyle{myStyle}{
    belowcaptionskip=1\baselineskip,
    breaklines=true,
    frame=none,
    numbers=none,
    basicstyle=\footnotesize\ttfamily,
    keywordstyle=\bfseries\color{green!40!black},
    commentstyle=\itshape\color{purple!40!black},
    identifierstyle=\color{blue},
    backgroundcolor=\color{gray!10!white},
    tabsize=2,
}



\lstset{style=myStyle}
\usepackage{array}
\usepackage{physics}

\begin{document}

\unitlength = .8mm


\section{The holographic duality}
We begin by reviewing the holographic duality, on top of which the MST conjecture is built. Start from the decoupling limit of the extremal black 1-brane solution of type IIB string theory, described in terms of the string frame metric, the dilaton $\Phi$, and the RR 2-form potential $C_2$ as
\ie\label{dsaiibdoness}
& ds_{\rm str}^2 = (\wt f_1(r))^{-{1\over 2}} (-dt^2 + dx^2) + (\wt f_1(r))^{1\over 2} (dr^2 + r^2 d\Omega_7^2),
\\
& e^\Phi = (\wt f_1(r))^{1\over 2},~~~~ C_2 =  \wt f_1^{-1} dt \wedge dx,
\\
& \wt f_1(r) = {c_1 N\over r^6},~~~~ c_1 = 32 \pi^2 g_B \ell_B^6 = 32\pi^2 g_B^{-{1\over 2}}M_{\rm pl}^{-6},
\fe
where $\ell_B$ is the type IIB string length, and $g_B$ is the type IIB string coupling (defined as the ratio between the F1 and D1 string tensions in the absence of RR axion and dilaton expectation value).
The standard holographic dictionary suggests an exact dual description in terms of the 2D ${\cal N}=(8,8)$ $U(N)$ SYM characterized by the gauge field $A_\mu$, adjoint scalar fields $\phi^i$, and adjoint fermions $\lambda_{\A +}$, $\lambda_{\da -}$. Here $i=1,\cdots,8$ is a vector index with respect to the $so(8)_R$ symmetry, and $\A, \da$ are chiral and anti-chiral spinor indices with respect to the $so(8)_R$. The Lorentzian action reads
\ie
S & = {1\over g_{\rm YM}^2} \int d^2x\, {\rm tr} \bigg( - {1\over 4} F_{\mu\nu} F^{\mu\nu} - {1\over 2} D_\mu \phi^i D^\mu \phi^i + {1\over 4} [\phi^i, \phi^j]^2
\\
&~~~~~~~~~~~~~~~~~~~ -  \lambda_{\A+} D_- \lambda_{\A+} -  \lambda_{\da -} D_+ \lambda_{\da-} - \lambda_{\A+}\C^i_{\A\da} [\phi^i, \lambda_{\da-} ] \bigg),
\fe
where the gauge coupling $g_{\rm YM}$ is identified as
\ie
g_{\rm YM}^2 = { g_B\over 2\pi \ell_B^2}
\fe
and $D_\mu \equiv\partial_\mu - i [A_\mu, \cdot]$ in the adjoint, with the trace taken in the fundamental.
Applying the S-duality transformation to (\ref{dsaiibdoness}) yields the purely (NS,NS) spacetime background\footnote{Note that the 3-form field strength $H_3 = dB_2 = e^{2\Phi} {6dr\over r}\wedge dt\wedge dx$ obeys $\wt g_B^{-2}\int_{S^7} e^{-2\Phi} * H = (2\pi \sqrt{\A'})^6 N$, which is the electric (NS,NS) flux sourced by $N$ fundamental strings.}
\ie{}
& d s_{\rm str}^2 =  (\wt f_1(r))^{-1} (-dt^2 + dx^2) + dr^2 + r^2 d\Omega_7^2,
\\
& e^{\Phi} = (\wt f_1(r))^{-{1\over 2}},~~~~ B_2 = \wt f_1^{-1} dt\wedge dx,
\\
& \wt f_1(r) = {c_1 N\over r^6},~~~~ c_1 = 32\pi^2 \wt g_B^{1\over 2} M_{\rm pl}^{-6} = 32\pi^2 \wt g_B^2 \wt\ell_B^6,
\fe
where $\wt g_B = g_B^{-1}$ and $\wt\ell_B = g_B^{1\over 2} \ell_B$ are the string coupling and length in the dual frame.


\printbibliography
\end{document}
\bye
